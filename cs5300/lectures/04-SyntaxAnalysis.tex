


\documentclass[usepdftitle=false,professionalfonts,compress ]{beamer}

%Packages to be included
\usepackage[latin1]{inputenc}
\usepackage{graphics,epsfig, subfigure}
\usepackage{url}
\usepackage[T1]{fontenc}
%\usepackage{listings}
\usepackage{hyperref}
\usepackage[english]{babel}

%%%%%%%%%%%%%%%%%%%%%%%%%%%%%%%%%%%%%%%%%%%%%%%%%
%%%%%%%%%% PDF meta data inserted here %%%%%%%%%%
%%%%%%%%%%%%%%%%%%%%%%%%%%%%%%%%%%%%%%%%%%%%%%%%%
\hypersetup{
	pdftitle={Syntax Analysis},
	pdfauthor={Kenneth Sundberg}
}





%%%%%% Beamer Theme %%%%%%%%%%%%%

\usetheme[]{Warsaw}
		
\title{Syntax Analysis}
\subtitle{CS 5300}
\author{Kenneth Sundberg}
\date{}






%%%%%%%%%%%%%%%%%%%%%%%%%%%%%%%%%%%%%%%%%%%%%%%%%
%%%%%%%%%% Begin Document  %%%%%%%%%%%%%%%%%%%%%%
%%%%%%%%%%%%%%%%%%%%%%%%%%%%%%%%%%%%%%%%%%%%%%%%%




\begin{document}
\frame[plain]{
	\frametitle{}
	\titlepage
	\vspace{-0.5cm}
	\begin{center}
	%\frontpagelogo
	\end{center}
}
\frame{
	\tableofcontents[hideallsubsections]
}






%%%%%%%%%%%%%%%%%%%%%%%%%%%%%%%%%%%%%%%%%   
%%%%%%%%%% Content starts here %%%%%%%%%%
%%%%%%%%%%%%%%%%%%%%%%%%%%%%%%%%%%%%%%%%%










\section{CFG}
		
\subsection{Context Free Grammars}

{
\begin{frame}\frametitle{Definitions}

	\begin{itemize}
	\item Terminals

	\begin{itemize}
	\item Basic symbols of language
			\item Tokens from scanner
				\end{itemize}

			\item Nonterminals

	\begin{itemize}
	\item Named groups of terminal and nonterminal symbols
			\item Impose structure on language
			\item One is the start symbol, traditionally listed first
				\end{itemize}

			\item Productions

	\begin{itemize}
	\item Rules of language
			\item Define Nonterminals meaning
			\item Express a rewriting relation between a single nonterminal and a sequence of terminals and nonterminal symbols
				\end{itemize}

				\end{itemize}

\end{frame}}





{
\begin{frame}\frametitle{Derivation}

	\begin{itemize}
	\item A \textit{sentence} is a sequence of terminals that can be produced by a grammar
			\item A sequence of terminals and nonterminals that can be produced by a grammer is in setential form
			\item Setential forms are produced by rewriting a nonterminal in a setential form using a production from the grammar
			\item The start symbol alone is in setential form
				\end{itemize}

\end{frame}}






{
\begin{frame}\frametitle{Leftmost and Rightmost}

	\begin{itemize}
	\item Derivations can replace any symbol
			\item Usually the leftmost or the rightmost symbol is chosen
			\item If always the leftmost then the sequence of productions is a leftmost derivation
				\end{itemize}

\end{frame}}





{
\begin{frame}\frametitle{Parse Trees}

	\begin{itemize}
	\item Represent derivations, without regard to replacement order
			\item A sentance with more than one parse tree is ambiguous
				\end{itemize}

\end{frame}}




{
\begin{frame}\frametitle{Stronger than RE}

	\begin{itemize}
	\item $a^nb^n$
			\item Supports nested structures
				\end{itemize}

\end{frame}}




{
\begin{frame}\frametitle{Example}

	\begin{itemize}
	\item E -> E + T
			\item E -> E - T
			\item E -> T
			\item T -> T * F
			\item T -> T / F
			\item T -> F
			\item F -> ( E )
			\item F -> id
				\end{itemize}

\end{frame}}










{
\begin{frame}\frametitle{Pumping Lemma}

	\begin{itemize}
	\item All context free langauges (L) have some number $p \geq 1$ such that:
			\item $\forall s = uvxyz \in L \rightarrow s = uv^nxy^nz \in L$
				\end{itemize}

\end{frame}}




{
\begin{frame}\frametitle{Example}

	\begin{itemize}
	\item Consider language $a^ib^ic^i$
			\item Consider string $a^pb^pc^p$
			\item vxy =

	\begin{itemize}
	\item $a^j$
			\item $a^jb^k$
			\item $b^j$
			\item $b^jc^k$
			\item $c^j$
				\end{itemize}

				\end{itemize}

\end{frame}}





{
\begin{frame}\frametitle{Programming Languages are not Context Free}

	\begin{itemize}
	\item $ L = {wcw | w \in (a|b)*}$
				\end{itemize}

\end{frame}}



\subsection{Transformations}

{
\begin{frame}\frametitle{Eliminating Left Recursion}

	\begin{itemize}
	\item A grammar is left recursive if it contains a production $A \rightarrow A\alpha$
			\item Replace $A_i \rightarrow A_j \gamma$ with $A_i \rightarrow \delta_1 \gamma | ... | \delta_n \gamma$
			\item Example

	\begin{itemize}
	\item $A\rightarrow Ac|Aad|bd|\epsilon$
				\end{itemize}

				\end{itemize}

\end{frame}}





{
\begin{frame}\frametitle{Left Factoring}

	\begin{itemize}
	\item Helps with predictive parsing
			\item Replace $A \rightarrow \alpha \beta_1 | \alpha \beta_2$
			\item With

	\begin{itemize}
	\item $A \rightarrow \alpha A'$
			\item $A' \rightarrow \beta_1 | \beta_2$
				\end{itemize}

				\end{itemize}

\end{frame}}





{
\begin{frame}\frametitle{Eliminating Ambiguity}

	\begin{itemize}
	\item $S \rightarrow i E t S$
			\item $S \rightarrow i E t S e S$
			\item $S \rightarrow a$
			\item $E \rightarrow b$
			\item Consider: i b t a e i b t a e a
				\end{itemize}

\end{frame}}








\section{Parsing}
		
\subsection{Top Down Parsing}

{
\begin{frame}\frametitle{Top Down Parsing}

	\begin{itemize}
	\item Staring from the root, construct a parse tree for a given string
			\item Equivalent to finding the leftmost derivation
			\item Can not handle left recursion
				\end{itemize}

\end{frame}}





{
\begin{frame}\frametitle{Recursive Descent}

	\begin{itemize}
	\item Simple way to write parser by hand
			\item Suitable for many languages (including C++)
			\item Each Nonterminal is a recursive function

	\begin{itemize}
	\item In the order of the production rule
			\item Nonterminals are called
			\item Terminals are consumed from input stream
				\end{itemize}

				\end{itemize}

\end{frame}}





\subsection{Bottom Up}

{
\begin{frame}\frametitle{Bottom Up Parsing}

	\begin{itemize}
	\item Staring from leaves construct a parse tree
			\item Equivalent to finding the rightmost derivation
			\item Commonly generate rightmost derivation in reverse (left to right)
			\item Generated parsers
				\end{itemize}

\end{frame}}






{
\begin{frame}\frametitle{LR(k) Parsing}

	\begin{itemize}
	\item Left to Right Parse
			\item Rightmost Derivation
			\item k-Token Look-ahead
			\item Most general algorithm (among practical choices)
				\end{itemize}

\end{frame}}






{
\begin{frame}\frametitle{LR(k) Grammar}

	\begin{itemize}
	\item Given Production $A \rightarrow B$
			\item Given Derivation step $\alpha A \gamma \Rightarrow \alpha B \gamma$
			\item Grammar only needs to scan $\alpha B$ and at most k tokens of $\gamma$
			\item Grammar can detect boundary between $B$ and $\gamma$
			\item Grammar can detect $B$
				\end{itemize}

\end{frame}}







{
\begin{frame}\frametitle{Model of LR(k) Parser}

	\begin{itemize}
	\item Input
			\item Stack -- Essential for PDA
			\item Parse Table -- Action and Goto portion
			\item Program -- Independent of language
				\end{itemize}

\end{frame}}






{
\begin{frame}\frametitle{Example}

	\begin{itemize}
	\item Grammar

	\begin{itemize}
	\item $S \rightarrow S'$
			\item $S \rightarrow aSbS$
			\item $S \rightarrow a$
				\end{itemize}

			\item Input

	\begin{itemize}
	\item aaababa
				\end{itemize}

			\item Table
			\end{itemize}
\begin{tabular}{|c|c|c|c|c|}
\hline
Item:& a  & b & \$ & S \\
\hline
0    & S2 &   &    & 1 \\
1    &    &   & acc&  \\
2    & S2 & R2& R2 & 3 \\
3    &    & S4  &    &  \\
4    & S2   &   &    & 5 \\
5    &    & R1  & R1   &  \\
\hline
\end{tabular}
				
\end{frame}}





{
\begin{frame}\frametitle{FIRST}

	\begin{itemize}
	\item FIRST($\alpha$) is the set of tokens that can be first in any derivation of $\alpha$
			\item Look at productions of $\alpha$

	\begin{itemize}
	\item If they begin with a terminal add it to FIRST($\alpha$)
			\item If they begin with a non-terminal $\beta$ add FIRST($\beta$) to FIRST($\alpha$)
				\end{itemize}

				\end{itemize}

\end{frame}}




{
\begin{frame}\frametitle{FOLLOW}

	\begin{itemize}
	\item FOLLOW($\alpha$) is the set of tokens that can follow any derivation of $\alpha$
			\item Used to determine which production to use based on look ahead
				\end{itemize}

\end{frame}}




{
\begin{frame}\frametitle{LR(0) Item sets}

	\begin{itemize}
	\item Set of productions with a dot
			\item $A -> \alpha \circ \beta$
			\item Means we are deriving A and have seen $\alpha$
				\end{itemize}

\end{frame}}





{
\begin{frame}\frametitle{LR(1) item sets}

	\begin{itemize}
	\item Augmented LR(0) items, also include next input token
				\end{itemize}

\end{frame}}



{
\begin{frame}\frametitle{Parse Table Construction}

	\begin{itemize}
	\item Start with $S' \rightarrow \dot S$
			\item If the dot is before a non-terminal  add items for all productions of that non-terminal
			\item Make new states by reading next symbol and adding transition
			\item Repeat until no new states are formed
				\end{itemize}

\end{frame}}






{
\begin{frame}\frametitle{Conflicts}

	\begin{itemize}
	\item Reduce - Reduce

	\begin{itemize}
	\item Grammar is not representable as table
				\end{itemize}

			\item Shift - Reduce

	\begin{itemize}
	\item Should be cleaned up, but Reduce takes priority
				\end{itemize}

			\item May be resolved by associativity
				\end{itemize}

\end{frame}}





\subsection{Error Handling}

{
\begin{frame}\frametitle{Types of Errors}

	\begin{itemize}
	\item Lexical

	\begin{itemize}
	\item Mistyped keywords
			\item Mismatched string delimiters
				\end{itemize}

			\item Syntactic

	\begin{itemize}
	\item Missing semicolons
			\item else without if
				\end{itemize}

			\item Semantic

	\begin{itemize}
	\item Type mismatch errors
			\item Context sensitive
				\end{itemize}

			\item Logical

	\begin{itemize}
	\item Run-time errors
			\item Program is valid but not intended
				\end{itemize}

				\end{itemize}

\end{frame}}






{
\begin{frame}\frametitle{Recovery Strategies}

	\begin{itemize}
	\item None

	\begin{itemize}
	\item Give up after first error
				\end{itemize}

			\item Panic-Mode

	\begin{itemize}
	\item Give up on current parse
			\item Discard tokens until a synchronizing token is found

	\begin{itemize}
	\item Typically a delimiter
				\end{itemize}

			\item Continue from synchronizing token
				\end{itemize}

				\end{itemize}

\end{frame}}




{
\begin{frame}\frametitle{Recovery Strategies}

	\begin{itemize}
	\item Phrase Level Corrections

	\begin{itemize}
	\item Insert or Delete tokens to correct parse
			\item Add semicolon
			\item Correct mis-spelled identifiers
			\item Must avoid infinite loops
				\end{itemize}

			\item Error Productions

	\begin{itemize}
	\item Augment grammar with anticipated erroneous constructs
			\item Generate appropriate error message when detected
				\end{itemize}

			\item Global Correction

	\begin{itemize}
	\item Of theoretical interest, not currently practical
			\item Can be used to inform phrase level recovery
			\item Given:

	\begin{itemize}
	\item Token string
			\item Grammer
				\end{itemize}

			\item Find:

	\begin{itemize}
	\item A Token string that the grammer parses with a minimal edit distance from the given string
				\end{itemize}

				\end{itemize}

				\end{itemize}

\end{frame}}









\end{document}